\section*{Foreword}

This book is a companion to the online book with the same name, a kind of "executive summary" of the main concepts and ideas within data structures and algorithms.
The online book also contains interactive explanations, visualisations and exercises, and can be reached from the following URL:
\begin{center}
  \url{\baseURL}
\end{center}
\marginQR{Visit the site}{Online book}{\baseURL}
The table of contents of the printed and online versions are the same, but sometimes there is additional information online which we chose not to include in this companion.
In these cases there are QR codes in the margin that will lead you directly to that section online.
For example, the QR code directly to the right will lead you to the main book website.


\subsection*{The OpenDSA project}

This book started out as a modernised and simplified version of a collection of teaching modules about data structures and algorithms from the open-source OpenDSA project, developed and maintained by Cliff Shaffer at Virginia Tech University.
That project is still alive and thriving, and is much more than just a book -- in their own words, it is\ldots

\begin{quotation}
  ``\ldots infrastructure and materials to support courses in a wide variety of Computer Science-related topics such as Data Structures and Algorithms (DSA), Formal Languages, Finite Automata, and Programming Languages.
  OpenDSA materials include many visualisations and interactive exercises. Our philosophy is that students learn best when they engage the material and then practice it until they have demonstrated their proficiency.''
\end{quotation}
\marginQR{Visit the project page}{OpenDSA}{https://opendsa-server.cs.vt.edu/}

OpenDSA is well worth a visit, regardless if you like our book or not:
\begin{center}
  \url{https://opendsa-server.cs.vt.edu/}
\end{center}

\subsection*{Acknowledgements}

Given that this is a live open source project, the texts in this book, as well as the visualisations and exercises in the online version, have been written and developed by a large number of people.
However, most of the credit goes to Cliff Shaffer, who together with his team is developing the OpenDSA project and website.

\newpage
